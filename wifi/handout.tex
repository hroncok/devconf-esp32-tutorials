% arara: lualatex
% arara: lualatex
% Typeset using lualatex

\documentclass{../tutorial}
\usepackage{wrapfig}

\title{Wi-Fi Communication with the \abbr{ESP32}}

\begin{document}

\begin{enumerate}

\item
    You will need a set-up environment and an \abbr{ESP32} board.

\section{Wi-Fi setup}

    The \abbr{ESP32} microcontroller evolved from the \abbr{ESP8266},
    which was designed as a wi-fi communication chip for other devices.
    It was widely used with Arduinos, for example.
    But then, people that the ESP8266 is, in many respects, more powerful than
    an Arduino, and started hacking on it directly.

    The \abbr{ESP8266} manufacturer, Espressiv Systems, caught on and created
    an even more powerful chip, this time tailored as a microcontroller.
    That's the \abbr{ESP32}.

    At its heart, it's still a Wi-Fi chip. Let's take advantage of that!


\item
    The conference's peculiar Wi-Fi setup is quite inconvenient for
    microcontrollers, but we can make our own Wi-Fi network!

    Turn the \abbr{ESP32}'s into an access point (`AP_IF`):

    \begin{lstlisting}[escapeinside=<>]
    import network
    ap = network.WLAN(network.AP_IF)
    ap.active(True)
    ap.config(essid='<\emph{«your own name»}>', password='<\emph{«something secret»}>', channel=<\emph{«some small number»}>)
    \end{lstlisting}

    \begin{comment}
        Ideally, run `nmcli d wifi` on your Linux computer to see occupied
        Wi-Fi channels, and pick one you don't see.
    \end{comment}

    The `ifconfig` method can give you interface parameters – IP address,
    subnet mask, gateway and DNS server:

    \begin{lstlisting}
    print(ap.ifconfig())
    \end{lstlisting}

    Now, connect to your Wi-Fi using your computer.
    Can you ping the device?

\item
    On your computer, figure out your new IP address using:

    \begin{lstlisting}[language=bash]
    ip -4 a
    \end{lstlisting}

    Keep it in mind for later.

\section{HTTP Requests}

\item
    Let's try to download a file from your computer to the \abbr{ESP32}
    using HTTP.

    On your computer, create an new directory and put a small text file in it.
    Then, `cd` to the directory and start a HTTP server on port 8000
    to share the file:

    \begin{lstlisting}[language=bash]
    python3 -m http.server
    \end{lstlisting}

    Meanwhile, in MicroPython, import $\mu$requests – a minimal
    implementation of Requests, Python's preferred HTTP client library:

    \begin{lstlisting}
    import urequests
    \end{lstlisting}

    Then, request your file!

    \begin{lstlisting}[escapeinside=<>]
    response = urequests.get('http://<\emph{«your.ip.address»}>:8000/<\emph{filename.txt}>')

    print(response.status_code)
    \end{lstlisting}

    If the status code is 200 OK, see what the response text is:

    \begin{lstlisting}
    print(response.text)
    \end{lstlisting}

\item
    Please return the gadgets when you're done experimenting.

\end{enumerate}
\end{document}

    \begin{lstlisting}
