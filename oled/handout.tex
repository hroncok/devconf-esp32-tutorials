% arara: lualatex
% arara: lualatex
% Typeset using lualatex

\documentclass{../tutorial}

\newcommand\defbitmap{
    \setlength{\fboxsep}{.05em}
    \lstset{literate=%
        {1}{{%
            \colorbox{black!60!white}{%
                \color{white}%
                \rule[-.1\baselineskip]{0pt}{.6\baselineskip}%
                1%
            }%
        }}{1}%
        {0}{{%
            \colorbox{white}{%
                \rule[-.1\baselineskip]{0pt}{.6\baselineskip}%
                0%
            }%
        }}{1}
        {0b}{{0b}}{2}
    }
}

\title{Drawing to an OLED display with MicroPython on \abbr{ESP32}}

\begin{document}

    This one is fun!
    (But honestly, you could just be using a computer monitor).

    You will need a set-up environment and an \abbr{ESP32} board
    with an I$^2$C OLED display.

\begin{enumerate}

\item
    Before you can use the display, you need a bit of setup code
    to specify the kind and size of the display, which pins it is
    connected to, and its I$^2$C address:

    \begin{lstlisting}
    from machine import Pin, I2C
    from ssd1306 import SSD1306_I2C

    i2c = I2C(scl=Pin(22, Pin.OUT), sda=Pin(21, Pin.OUT))

    oled = SSD1306_I2C(width=128, height=64, i2c=i2c, addr=0x3c)
    \end{lstlisting}

\item
    To draw graphics, MicroPython keeps a copy of the screen's contents in
    memory in a `FrameBuffer`, which provides simple graphics commands
    such as `line`, `rect` and `text` to manipulate the buffer.
    The `SSD1306_I2C` driver has this API plus a `show` command to send
    the saved picture to the screen.

    \begin{lstlisting}
    oled.pixel(10, 20, 1)               # x, y, color
    oled.line(0, 0, 64, 64, 1)          # x1, y1, x2, y2, color
    oled.text("Hello World", 0, 0, 1)   # text, x, y, color
    oled.rect(110, 2, 10, 5, 1)         # x, y, width, height, color
    oled.fill_rect(100, 20, 10, 5, 1)   # x, y, width, height, color
    oled.show()
    \end{lstlisting}

\item
    You can also use a `FrameBuffer` with a chunk of memory,
    like Python's `bytearray`, and copy (`blit`) it into another `FrameBuffer`,
    like the screen.

    You can define a monochrome bitmap with `0`s and `1`s using Python's
    binary literals.
    What's not always straightforward is matching the bits up with the
    framebuffer's parameters.
    Try this bitmap with a height and width of 8\,px, and the format
    `MONO_HLSB` – Monochrome, Horizontal, Least significant Bit first:

    {
    \defbitmap
    \begin{lstlisting}
    heart = [
        0b00000000,
        0b00110110,
        0b01111111,
        0b01111111,
        0b00111110,
        0b00011100,
        0b00001000,
        0b00000000,
    ]
    \end{lstlisting}
    }
    \begin{lstlisting}
    import framebuf
    icon = framebuf.FrameBuffer(bytearray(heart), 8, 8, framebuf.MONO_HLSB)

    oled.blit(icon, 0, 0)
    oled.show()
    \end{lstlisting}

\item
    Or try to write a wider bitmap:

    {
    \defbitmap
    \begin{lstlisting}[escapechar=!]
    invader = [
        0b00000100, 0b00010000,
        0b00000010, 0b00100000,
        0b00000111, 0b11110000,
        0b00001101, 0b11011000,
        0b00011111, 0b11111100,
        0b00010111, 0b11110100,
        0b00010100, 0b00010100,
        0b00000011, 0b01100000,
    ]
    \end{lstlisting}
    }
    \begin{lstlisting}
    sprite = framebuf.FrameBuffer(bytearray(invader), 16, 8, framebuf.MONO_HLSB)
    oled.blit(sprite, 0, 16)
    oled.show()
    \end{lstlisting}

\item
    Let your imagination run free!

\item
    Please return the board when you're done experimenting.

\end{enumerate}
\end{document}
