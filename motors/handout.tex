% arara: lualatex
% arara: lualatex
% Typeset using lualatex

\documentclass{../tutorial}
\usepackage{wrapfig}

\title{Motors with MicroPython on \abbr{ESP32}}

\begin{document}

\begin{enumerate}

\item
    You will need a set-up environment and an \abbr{ESP32} board
    with some motors and batteries.

\section{Motors and Electricity}

\item
    A \emph{DC} (direct current) electric motor turns when you provide
    electricity – voltage difference – on its two inputs.

    Unlike servos and stepper motors, the speed or position of a plain DC
    motor can't be controlled precisely.
    (A~common problem for wheeled robots is figuring out how far they've driven
    and where they are.)

    Our motors need quite a lot more current (and voltage) than the \abbr{ESP32}
    microcontroller, so they need a separate power source – batteries.

    To drive the motors using the \abbr{ESP32}'s low-current (but precisely
    controlled) pins, we'll use the \abbr{L293D} chip, which contains two
    circuits called \emph{H~bridges}.

    Software-wise, the the \abbr{L293D} makes it seem as if the motors
    were connected between the \abbr{ESP32}'s pins (\#26 and \#12 for motor A,
    \#14 and \#27 for motor B).
    But additionally, there's a pin to enable or fully disable each motor
    (\#25 for A, \#13 for B).

\item
    Turn on motor A:

    \begin{lstlisting}
    from machine import Pin

    a_enable = Pin(25, Pin.OUT)
    a1 = Pin(26, Pin.OUT)
    a2 = Pin(12, Pin.OUT)

    a1.value(0)
    a2.value(1)
    a_enable.value(1)
    \end{lstlisting}

\item
    The `a1` and `a2` pins are “connected” to the motor's terminals.
    When one's at `1` and the other at `0`, there's a voltage difference,
    and the motor turns.

    What happens when the difference goes the other way?

    \begin{lstlisting}
    a1.value(1)
    a2.value(0)
    \end{lstlisting}

\item
    What happens when there's no difference?

    \begin{lstlisting}
    a1.value(1)
    a2.value(1)
    \end{lstlisting}

\item
    Does disabling the whole thing work?

    \begin{lstlisting}
    a_enable.value(0)
    \end{lstlisting}

\item
    The enable pin is suited for being switched on and off rapidly.

    Try using PWM to control the motor's speed like you'd control an
    LED's brightnes.

\item
    Turn on the other motor. Check if it works similarly.

\item
    There should be a MicroPython-enabled car around.
    Borrow it and make it dance!

\item
    If you know how to communicate via Wi-Fi,
    try controlling the motors remotely.

\item
    Please return the gadgets when you're done experimenting.

\end{enumerate}
\end{document}

    \begin{lstlisting}
