% arara: lualatex
% arara: lualatex
% Typeset using lualatex

\documentclass{../tutorial}
\usepackage{wrapfig}

\title{Measuring Temperature with MicroPython on the \abbr{ESP32}}

\begin{document}

    One of the cheapest ways to measure temperature is the \abbr{DS18S20}
    digital thermometer using a one-wire protocol.

    You will need a set-up environment and an \abbr{ESP32} board with
    this thermometer.

\begin{enumerate}

\item
    If the thermometer is not connected, connect it to \abbr{DEV1} (pin \#32)
    and the power and ground pins around it.

    Make \emph{extra sure} the round side is facing \emph{away} from the
    \abbr{ESP32}.

\item
    Like with I$^2$C, a driver for the \abbr{DS18S20}'s protocol is included
    in the MicroPython firmware.
    Actually, there are two drivers: one for the wire protocol, and one for the
    specific thermometer.
    Import them both, and scan for connected thermometers.
    You should only get one, though the protocol supports connecting
    several thermometers to one wire.

    \begin{lstlisting}
    from machine import Pin
    import onewire
    from ds18x20 import DS18X20

    pin = Pin(32, Pin.IN)
    ds = DS18X20(onewire.OneWire(pin))
    sensors = ds.scan()
    print(sensors)
    \end{lstlisting}

\item
    To read the temperature, you'll need to initiate the measurement with
    a command called `convert_temp`, then wait at least $750 \si{\milli\second}$
    \footnote{There is a polling \abbr{API} too, but waiting works well enough.}
    for processing before reading the data.

    The device gives temperature readouts in $\si{\celsius}$.

    \begin{lstlisting}
    from time import sleep_ms

    ds.convert_temp()
    sleep_ms(750)
    temperature = ds.read_temp(sensors[0])
    print("It's", temperature, 'degrees C')
    \end{lstlisting}


\item
    Hold the sensor in your fingers to heat it up a bit, then
    measure the temperature again.


\item
    If you already know how to light stuff up or make stuff turn,
    indicate the temperature using that stuff!


\item
    Please return the gadgets when you're done experimenting.

\end{enumerate}
\end{document}

    \begin{lstlisting}
